\documentclass{beamer}

\usefonttheme{professionalfonts} % using non standard fonts for beamer
\usefonttheme{serif} % default family is serif

\usepackage{hyperref}

%\usepackage{minted}

\usepackage{animate}

\usepackage{graphicx}

\def\Put(#1,#2)#3{\leavevmode\makebox(0,0){\put(#1,#2){#3}}}

\usepackage{color}

\usepackage{tikz}

\usepackage{amssymb}

\usepackage{enumerate}

\newcommand\blfootnote[1]{%

  \begingroup

  \renewcommand\thefootnote{}\footnote{#1}%

  \addtocounter{footnote}{-1}%

  \endgroup

}

\makeatletter

%%%%%%%%%%%%%%%%%%%%%%%%%%%%%% Textclass specific LaTeX commands.

 % this default might be overridden by plain title style

 \newcommand\makebeamertitle{\frame{\maketitle}}%

 % (ERT) argument for the TOC

 \AtBeginDocument{%

   \let\origtableofcontents=\tableofcontents

   \def\tableofcontents{\@ifnextchar[{\origtableofcontents}{\gobbletableofcontents}}

   \def\gobbletableofcontents#1{\origtableofcontents}

 }

%%%%%%%%%%%%%%%%%%%%%%%%%%%%%% User specified LaTeX commands.

\usetheme{Malmoe}

% or ...

\useoutertheme{infolines}

\addtobeamertemplate{headline}{}{\vskip2pt}

\setbeamertemplate{theorems}[numbered]

\setbeamercovered{transparent}

% or whatever (possibly just delete it)

\makeatother

\begin{document}
\title[Discussion 5 - Recurrences]{CS/MATH 111, Discrete Structures - Fall 2018. \\ Discussion 5 - Linear Recurrence Relations }
\author[CS111]{Andres, Sara, Elena}
\institute[Fall'18]{University of California, Riverside}
\makebeamertitle
\newif\iflattersubsect

\AtBeginSection[] {
    \begin{frame}<beamer>
    \frametitle{Outline} 
    \tableofcontents[currentsection]  
    \end{frame}
    \lattersubsectfalse
}

\AtBeginSubsection[] {
    \begin{frame}<beamer>
    \frametitle{Outline} 
    \tableofcontents[currentsubsection]  
    \end{frame}
}

\section{Fibonacci grows exponentially}

\begin{frame}{Fibonacci recurrence}
    \begin{itemize}
        \item Fibonnacci numbers / Fibonnacci sequence
        \item First two and subsequent numbers:
        \begin{itemize}
            \item $F_0 = 1$
            \item $F_1 = 1$
            \item $F_n = F_{n-1} + F_{n-2}$ {\scriptsize, when $n \geq 2$.}
        \end{itemize}
        \item Fibonacci grows exponentially with $n$.
        \item Prove that: $$0.5 \cdot 1.5^n \leq F_n \leq 2^n$$
      \end{itemize}
\end{frame}

\begin{frame}{Fibonacci recurrence}
Proof by induction using $F_0 = 1, F_1 = 1, \ldots$:
\begin{equation}\label{eq1}
0.5 \cdot 1.5^n \leq F_n \leq 2^n
\end{equation}
    \begin{enumerate}
        \item Base case: 
        \begin{itemize}
         \item $n=0, F_0=1: 0.5 \cdot 1.5^0 \leq 1 \leq 2^0 = 0.50 \leq 1 \leq 1$.
         \item $n=1, F_1=1: 0.5 \cdot 1.5^1 \leq 1 \leq 2^1 = 0.75 \leq 1 \leq 2$.
         \item $n=2, F_2=2: 0.5 \cdot 1.5^2 \leq 1 \leq 2^2 = 1.125 \leq 2 \leq 4$.
         \\ $\vdots$
        \end{itemize}
        \item Assumption step:
        \begin{itemize}
            \item Assume \eqref{eq1} holds for all $n \leq k - 1$.
        \end{itemize}
        \item Induction step:
        \begin{itemize}
            \item Prove that \eqref{eq1} holds for all $n \leq k$.
        \end{itemize}
    \end{enumerate}
\end{frame}

\begin{frame}{Fibonacci recurrence}
Proof by induction using $F_0 = 1, F_1 = 1, \ldots$:
$$0.5 \cdot 1.5^n \leq F_n \leq 2^n$$
    \begin{enumerate}[3]
        \item Induction step:
        \begin{itemize}
            \item Prove that \eqref{eq1} holds for all $n \leq k$.
            \begin{enumerate}[1)]
             \item $ F_k \leq 2^k $ \\
                $ F_k = F_{k-1} + F_{k-2} $ \\
                $ F_k \leq 2^{k-1} + 2^{k-2} $ {\scriptsize (by assumption.)} \\
                $ F_k = 2^{k-2} \cdot (2 + 1) $ \\
                $ F_k = 2^{k-2} \cdot 3 $ \\
                $ F_k \leq 2^{k-2} \cdot 4 $ \\
                $ F_k = 2^{k-2} \cdot 2^2 $ \\
                $ F_k = 2^k $ \\
                \vspace{0.3cm}
                $ F_n = \mathcal{O}(2^n) $
            \end{enumerate}
        \end{itemize}
    \end{enumerate}
\end{frame}

\begin{frame}{Fibonacci recurrence}
Proof by induction using $F_0 = 1, F_1 = 1, \ldots$:
$$0.5 \cdot 1.5^n \leq F_n \leq 2^n$$
    \begin{enumerate}[3]
        \item Induction step:
        \begin{itemize}
            \item Prove that \eqref{eq1} holds for all $n \leq k$.
            \begin{enumerate}[2)]
             \item $ F_k \geq \frac{1}{2} \cdot 1.5^k $ \\
                $ F_k = F_{k-1} + F_{k-2} $ \\
                $ F_k \geq 1.5^{k-1} + 1.5^{k-2} $ {\scriptsize (by assumption.)} \\
                $ F_k = 1.5^{k-2} \cdot (1.5 + 1) $ \\
                $ F_k = 1.5^{k-2} \cdot 2.5 $ \\
                $ F_k \geq 1.5^{k-2} \cdot 2.25 $ \\
                $ F_k = 1.5^{k-2} \cdot 1.5^2 $ \\
                $ F_k = 1.5^k $ \\
                \vspace{0.3cm}
                $ F_n = \Omega (1.5^n) $
            \end{enumerate}
        \end{itemize}
    \end{enumerate}
\end{frame}

\begin{frame}{Fibonacci recurrence}
    $$ 0.5 \cdot 1.5^n \leq F_n \leq 2^n $$
    \vspace{0.2cm}
    $$ F_n = \mathcal{O}(2^n) $$
    $$ F_n = \Omega (1.5^n) $$
    $$ is \ F_n = \Theta (\ ) \ ? $$
\end{frame}

\section{Fibonnacci numbers}

\begin{frame}{Fibonacci numbers}
    \begin{itemize}
    \item Let's rewrite our recurrence following our previous notation:
    \begin{equation}\label{fibo}
        F_n = F_{n-1} + F_{n-2}. 
    \end{equation} 
    {\footnotesize for $n \geq 2$.}
        \begin{itemize}
            \item $F_0 = 1$ 
            \item $F_1 = 1$
        \end{itemize}
    \item Since $F_n$ grows exponentially, we will assume:
        \begin{equation}\label{eq0}
            F_n = x^n 
        \end{equation}
    \item Plugging \eqref{eq0} into \eqref{fibo}:
        $$ x^n = x^{n-1} + x^{n-2} $$
        after dividing by $x^{n-2}$: 
        $$ x^2 - x - 1 = 0 $$
    \end{itemize}
\end{frame}

\begin{frame}{Fibonacci numbers}
    $$ x^2 - x - 1 = 0 $$
    \begin{itemize}
     \item This equation has two roots: $x_1 = \frac{1}{2}(1 + \sqrt{5})$ and $x_2 = \frac{1}{2}(1 - \sqrt{5})$.
     \item $x_1$ is the \textit{golden ratio}\footnote{\scriptsize \url{https://en.wikipedia.org/wiki/Golden_ratio}} $\phi \approx 1.618$ and $x_2 = 1 - \phi \approx -0.618$.
     \item Do they satisfy \eqref{fibo}? It works for $n = 0$ but not for $n = 1$.
     \item Works for the main recurrence but not for the initial conditions...
    \end{itemize}
\end{frame}

\begin{frame}{Fibonacci numbers}
\begin{theorem}\label{theo:1}
Let $c_1$ and $c_2$ be real numbers. Suppose that $r^2 - c_1 r - c_2 = 0$ has two distinct roots $r_1$ and $r_2$. Then the sequence ${a_n}$ is a solution of the recurrence relation $a_n = c_1 a_{n−1} + c_2 a_{n−2}$ if and only if $$a_n = \alpha_1 r_1^n + \alpha_2 r_2^n$$ for $n = 0,1,2,\ldots$, where $\alpha_1$ and $\alpha_2$ are constants\footnote{\scriptsize Proof available at [Rosen, 2015. pg 515].}.
\end{theorem}
\end{frame}

\begin{frame}{Fibonacci numbers}
    $$ x^2 - x - 1 = 0 $$
    \begin{itemize}
     \item This equation has two roots: $x_1 = \frac{1}{2}(1 + \sqrt{5})$ and $x_2 = \frac{1}{2}(1 - \sqrt{5})$.
     \item Therefore by Theorem \ref{theo:1} it follows that the Fibonacci numbers are given by:
     $$ F_n = \alpha_1 x_1^n + \alpha_2 x_2^n $$
     \item This form is called the \textit{general form of the solution}.
    \end{itemize}
\end{frame}

\begin{frame}{Fibonacci numbers}
     $$ F_n = \alpha_1 x_1^n + \alpha_2 x_2^n $$
    \begin{itemize}
        \item Plugging the initial conditions into this equation we will get a system of two equation and two parameters:
        $$ \alpha_1 x_1^0 + \alpha_2 x_2^0 = 1 $$
        $$ \alpha_1 x_1^1 + \alpha_2 x_2^1 = 1 $$
        After substituting:
        $$ \alpha_1 + \alpha_2 = 1 $$
        $$ \alpha_1 \cdot \frac{1}{2}(1 + \sqrt{5}) + \alpha_2 \cdot \frac{1}{2}(1 - \sqrt{5}) = 1 $$
    \end{itemize}
\end{frame}

\begin{frame}{Fibonacci numbers}
        $$ \alpha_1 + \alpha_2 = 1 $$
        $$ \alpha_1 \cdot \frac{1}{2}(1 + \sqrt{5}) + \alpha_2 \cdot \frac{1}{2}(1 - \sqrt{5}) = 1 $$
    \begin{itemize}
        \item Solving the system, we get:
        $$ \alpha_1 = \frac{\sqrt{5} + 1}{2 \sqrt{5}} $$
        $$ \alpha_2 = \frac{\sqrt{5} - 1}{2 \sqrt{5}} $$
    \end{itemize}
\end{frame}

\begin{frame}{Fibonacci numbers}
    \begin{itemize}
        \item This give us a solution for $F_n$:
        $$ F_n = \frac{\sqrt{5} + 1}{2 \sqrt{5}} \Bigg(\frac{1 + \sqrt{5}}{2} \Bigg)^n + \frac{\sqrt{5} - 1}{2 \sqrt{5}} \Bigg(\frac{1 - \sqrt{5}}{2} \Bigg)^n $$
        Simplified as:
        $$ F_n = \frac{1}{\sqrt{5}} \Bigg[ \Bigg(\frac{1 + \sqrt{5}}{2} \Bigg)^{n+1} - \Bigg(\frac{1 - \sqrt{5}}{2} \Bigg)^{n+1} \Bigg] $$
    \end{itemize}
\end{frame}

\begin{frame}{Fibonacci numbers}
    $$ F_n = \frac{1}{\sqrt{5}} \Bigg[ \Bigg(\frac{1 + \sqrt{5}}{2} \Bigg)^{n+1} - \Bigg(\frac{1 - \sqrt{5}}{2} \Bigg)^{n+1} \Bigg] $$
    \begin{itemize}
        \item Note that when $n \rightarrow \infty$:
        $$ F_n \approx \frac{1}{\sqrt{5}} \Bigg(\frac{1 + \sqrt{5}}{2} \Bigg)^{n+1} $$
    \end{itemize}
    \begin{figure}
    \centering
    \def\svgwidth{0.45\columnwidth}
    \input{fibo.pdf_tex}
    \end{figure}
\end{frame}

\section*{Multiplicity in recurrences}

\begin{frame}{Double root}
    Let's have the following recurrence relation:
        $$ a_n = 4 a_{n-1} - 4 a_{n-2} $$
    with initial condition $a_0 = 1$ and $a_1 = 3$.
    \begin{itemize}
     \item Let's find the characteristic equation and its roots: \\
     $ x^2 - 4x + 4 = 0 $. \\
     $ (x - 2)^2 = 0 $. \\
     So, $x_{1,2} = 2$.  \hspace{1cm} {\scriptsize [``double root'' or a root with multiplicity of 2.]}
    \end{itemize}
\end{frame}

\begin{frame}{Double root}
    Let's have the following recurrence relation:
        $$ a_n = 4 a_{n-1} - 4 a_{n-2} $$
    with initial condition $a_0 = 1$ and $a_1 = 3$.
    \begin{itemize}
     \item All functions $\alpha_1 2^n$ are good candidates, but...
     \item ... we need our solution to be parametrized by \textbf{two} parameters.
     \item We can try the function $n 2^n$:
     $$ n 2^n = 4(n -1) 2^{n-1} - 4 (n - 2) 2^{n-2} $$
     by simple algebra we see that is is indeed true.
    \end{itemize}
\end{frame}

\begin{frame}{Double root}
    Let's have the following recurrence relation:
        $$ a_n = 4 a_{n-1} - 4 a_{n-2} $$
    with initial condition $a_0 = 1$ and $a_1 = 3$.
    \begin{itemize}
     \item Since $n 2^n$ is a solution, so is any function $\alpha_2 n 2^n$.
     \item We can combine the two types of solutions in a general form:
     $$ a_n = \alpha_1 2^n + \alpha_2 n 2^n $$
     then we can continue...
    \end{itemize}
\end{frame}

\begin{frame}{Double root}
    Let's have the following recurrence relation:
        $$ a_n = 4 a_{n-1} - 4 a_{n-2} $$
    with initial condition $a_0 = 1$ and $a_1 = 3$.
    \begin{itemize}
     \item Initial condition equations and their solutions:
     $$ \alpha_1 \cdot 2^0 + \alpha_2 \cdot 0 \cdot 2^0 = 1 $$
     $$ \alpha_1 \cdot 2^1 + \alpha_2 \cdot 1 \cdot 2^1 = 3 $$
     which reduce to:
     $$ \alpha_1 = 1 $$
     $$  2 \alpha_1 + 2 \alpha_2 = 3 $$
    \end{itemize}
\end{frame}

\begin{frame}{Double root}
    Let's have the following recurrence relation:
        $$ a_n = 4 a_{n-1} - 4 a_{n-2} $$
    with initial condition $a_0 = 1$ and $a_1 = 3$.
    \begin{itemize}
     \item Initial condition equations and their solutions:
     $$ \alpha_1 = 1 $$
     $$  2 \alpha_1 + 2 \alpha_2 = 3 $$
     We get $\alpha_1 = 1$ and $\alpha_2 = \frac{1}{2}$.
    \end{itemize}
\end{frame}

\begin{frame}{Double root}
    Let's have the following recurrence relation:
        $$ a_n = 4 a_{n-1} - 4 a_{n-2} $$
    with initial condition $a_0 = 1$ and $a_1 = 3$.
    \begin{itemize}
     \item Final answer:
     $$ a_n = 2^n + \frac{1}{2} n 2^n $$
    \end{itemize}
\end{frame}

\section{Homogeneous Recurrence Equations}

\setbeamercovered{invisible}
\begin{frame}{Example 1}
    Solve the recurrence relation $a_n = a_{n-1} + 2 a_{n-2}$ with initial conditions $a_0 = 2$ and $a_1 = 7$.
    \footnotesize
    \begin{enumerate}[<+->]
        \item Characteristic equation and its roots: \\
        $ x^2 - x - 2 = 0 $ \\
        $ (x + 1)(x - 2) = 0 $ \\
        So, $x_1 = 2$ and $x_2 = -1$.
        \item General form of the solution: \\
        $ a_n = \alpha_1 2^n + \alpha_2 (-1)^n $.
        \item Initial condition equations and their solutions: \\
        $ a_0 = \alpha_1 + \alpha_2 = 2 $ \\
        $ a_1 = \alpha_1 2 + \alpha_2 (-1) = 7 $ \\
        So, $\alpha_1 = 3$ and $\alpha_2 = -1$.
        \item Final answer: \\
        $ a_n = 3 \cdot 2^n - (-1)^n $ is a solution.
    \end{enumerate}
\end{frame}

\begin{frame}{Example 2}
    What is the solution of the recurrence relation $a_n = -a_{n-1} + 4 a_{n-2} + 4 a_{n-3}$ with initial conditions $a_0 = 8$, $a_1 = 6$ and $a_2 = 26$.
    \footnotesize
    \begin{enumerate}[<+->]
        \item Characteristic equation and its roots: \\
        $ x^3 + x^2 - 4x - 4 = 0 $ \\
        $ (x + 1)(x + 2)(x - 2) = 0 $ \\
        So, $x_1 = -1$, $x_2 = -2$ and $x_3 = 2$.
        \item General form of the solution: \\
        $ a_n = \alpha_1 (-1)^n + \alpha_2 (-2)^n + \alpha_3 2^n $.
        \item Initial condition equations and their solutions: \\
        $ a_0 = \alpha_1 + \alpha_2 + \alpha_3 = 8 $ \\
        $ a_1 = - \alpha_1 - 2 \alpha_2 + 2 \alpha_3 = 6 $ \\
        $ a_2 = \alpha_1 + 4 \alpha_2 + 4 \alpha_3 = 26 $ \\
        So, $\alpha_1 = 2$, $\alpha_2 = 1$ and $\alpha_3 = 5$.
        \item Final answer: \\
        $ a_n = 2 \cdot (-1)^n + (-2)^n + 5 \cdot 2^n $ is a solution.
    \end{enumerate}
\end{frame}

\section{Word problem}

\setbeamercovered{transparent}
\begin{frame}{Word problem}
    \begin{enumerate}[<+->]
        \item Find a recurrence relation for the number of ways to climb $n$ stairs if the person climbing the stairs can take one stair or two stairs at a time.
        \item What are the initial conditions?
        \item In how many ways can this person climb a flight of eight stairs?
    \end{enumerate}
\end{frame}

\setbeamercovered{invisible}
\begin{frame}{Word problem}
    Let $S_n$ denote the number of ways of climbing the stairs.
    \begin{enumerate}[<+->]
        \item Let $n \geq 3$.  The last step either was a single step, for which there are $S_{n-1}$ possibilities, or a double step, for which there are $S_{n-2}$ possibilities.  The recurrence is: $ S_n = S_{n-1} + S_{n-2} $ for $n \geq 3$.
        \item Whe have $S_1 = 1$ and $S_2 = 2$.  You can take two stairs either directly or by taking a stair at a time.
        \item The recurrence gives the Fibonacci sequence:
        $$ 1,2,3,5,8,13,21,34,55,\ldots $$
        Hence there are $S_8 = 34$ ways to climb a flight of eight stairs.
    \end{enumerate}
\end{frame}

\section*{Bibliography}

\begin{frame}{Bibliography}
    \begin{itemize}
        \item Discrete Mathematics and Its Applications. Rosen, K.H. 2012. McGraw-Hill. \\
        Chapter 8: Advanced Counting Techniques. \\
        Section 8.2: Solving Linear Recurrence Relations.
    \end{itemize}
\end{frame}


\end{document}
