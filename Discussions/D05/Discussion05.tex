\documentclass{beamer}

\usepackage{hyperref}

%\usepackage{minted}

\usepackage{animate}

\usepackage{graphicx}

\def\Put(#1,#2)#3{\leavevmode\makebox(0,0){\put(#1,#2){#3}}}

\usepackage{color}

\usepackage{tikz}

\usepackage{amssymb}

\usepackage{enumerate}


\newcommand\blfootnote[1]{%

  \begingroup

  \renewcommand\thefootnote{}\footnote{#1}%

  \addtocounter{footnote}{-1}%

  \endgroup

}

\makeatletter

%%%%%%%%%%%%%%%%%%%%%%%%%%%%%% Textclass specific LaTeX commands.

 % this default might be overridden by plain title style

 \newcommand\makebeamertitle{\frame{\maketitle}}%

 % (ERT) argument for the TOC

 \AtBeginDocument{%

   \let\origtableofcontents=\tableofcontents

   \def\tableofcontents{\@ifnextchar[{\origtableofcontents}{\gobbletableofcontents}}

   \def\gobbletableofcontents#1{\origtableofcontents}

 }

%%%%%%%%%%%%%%%%%%%%%%%%%%%%%% User specified LaTeX commands.

\usetheme{Malmoe}

% or ...

\useoutertheme{infolines}

\addtobeamertemplate{headline}{}{\vskip2pt}



\setbeamercovered{transparent}

% or whatever (possibly just delete it)

\makeatother

\begin{document}
\title[Discussion 5 - Recurrences]{CS/MATH 111, Discrete Structures - Fall 2018. \\ Discussion 5 - Linear Recurrence Relations }
\author[CS111]{Andres, Sara, Elena}
\institute[Fall'18]{University of California, Riverside}
\makebeamertitle
\newif\iflattersubsect

\AtBeginSection[] {
    \begin{frame}<beamer>
    \frametitle{Outline} 
    \tableofcontents[currentsection]  
    \end{frame}
    \lattersubsectfalse
}

\AtBeginSubsection[] {
    \begin{frame}<beamer>
    \frametitle{Outline} 
    \tableofcontents[currentsubsection]  
    \end{frame}
}

\section{Fibbonacci recurrence}

\begin{frame}{Fibbonacci recurrence}
    \begin{itemize}
        \item Fibonnacci numbers / Fibonnacci sequence
        \item First two and subsequent numbers:
        \begin{itemize}
            \item $F_0 = 1$
            \item $F_1 = 1$
            \item $F_n = F_{n-1} + F_{n-2}$ {\scriptsize, when $n \geq 2$.}
        \end{itemize}
        \item Fibbonacci grows exponentially with $n$.
        \item Prove that: $$0.5 \cdot 1.5^n \leq F_n \leq 2^n$$
      \end{itemize}
\end{frame}

\begin{frame}{Fibbonacci recurrence}
Proof by induction using $F_0 = 1, F_1 = 1, \ldots$:
\begin{equation}\label{eq1}
0.5 \cdot 1.5^n \leq F_n \leq 2^n
\end{equation}
    \begin{enumerate}
        \item Base case: 
        \begin{itemize}
         \item $n=0, F_0=1: 0.5 \cdot 1.5^0 \leq 1 \leq 2^0 = 0.50 \leq 1 \leq 1$.
         \item $n=1, F_1=1: 0.5 \cdot 1.5^1 \leq 1 \leq 2^1 = 0.75 \leq 1 \leq 2$.
         \item $n=2, F_2=2: 0.5 \cdot 1.5^2 \leq 1 \leq 2^2 = 1.125 \leq 2 \leq 4$.
         \\ $\vdots$
        \end{itemize}
        \item Assumption step:
        \begin{itemize}
            \item Assume \eqref{eq1} holds for all $n \leq k - 1$.
        \end{itemize}
        \item Induction step:
        \begin{itemize}
            \item Prove that \eqref{eq1} holds for all $n \leq k$.
        \end{itemize}
    \end{enumerate}
\end{frame}

\begin{frame}{Fibbonacci recurrence}
Proof by induction using $F_0 = 1, F_1 = 1, \ldots$:
$$0.5 \cdot 1.5^n \leq F_n \leq 2^n$$
    \begin{enumerate}[3]
        \item Induction step:
        \begin{itemize}
            \item Prove that \eqref{eq1} holds for all $n \leq k$.
            \begin{enumerate}[1)]
             \item $ F_k \leq 2^k $ \\
                $ F_k = F_{k-1} + F_{k-2} $ \\
                $ F_k \leq 2^{k-1} + 2^{k-2} $ {\scriptsize (by assumption.)} \\
                $ F_k = 2^{k-2} \cdot (2 + 1) $ \\
                $ F_k = 2^{k-2} \cdot 3 $ \\
                $ F_k \leq 2^{k-2} \cdot 4 $ \\
                $ F_k = 2^{k-2} \cdot 2^2 $ \\
                $ F_k = 2^k $ \\
                \vspace{0.3cm}
                $ F_n = \mathcal{O}(2^n) $
            \end{enumerate}
        \end{itemize}
    \end{enumerate}
\end{frame}

\begin{frame}{Fibbonacci recurrence}
Proof by induction using $F_0 = 1, F_1 = 1, \ldots$:
$$0.5 \cdot 1.5^n \leq F_n \leq 2^n$$
    \begin{enumerate}[3]
        \item Induction step:
        \begin{itemize}
            \item Prove that \eqref{eq1} holds for all $n \leq k$.
            \begin{enumerate}[2)]
             \item $ F_k \geq \frac{1}{2} \cdot 1.5^k $ \\
                $ F_k = F_{k-1} + F_{k-2} $ \\
                $ F_k \geq 1.5^{k-1} + 1.5^{k-2} $ {\scriptsize (by assumption.)} \\
                $ F_k = 1.5^{k-2} \cdot (1.5 + 1) $ \\
                $ F_k = 1.5^{k-2} \cdot 2.5 $ \\
                $ F_k \geq 1.5^{k-2} \cdot 2.25 $ \\
                $ F_k = 1.5^{k-2} \cdot 1.5^2 $ \\
                $ F_k = 1.5^k $ \\
                \vspace{0.3cm}
                $ F_n = \Omega (1.5^n) $
            \end{enumerate}
        \end{itemize}
    \end{enumerate}
\end{frame}

\begin{frame}{Fibbonacci recurrence}
    $$ 0.5 \cdot 1.5^n \leq F_n \leq 2^n $$
    \vspace{0.2cm}
    $$ F_n = \mathcal{O}(2^n) $$
    $$ F_n = \Omega (1.5^n) $$
    $$ is \ F_n = \Theta (\ ) \ ? $$
\end{frame}

\section{Homogeneous Recurrence Equations}

\setbeamercovered{invisible}
\begin{frame}{Example 1}
    Solve the recurrence relation $a_n = a_{n-1} + 2 a_{n-2}$ with initial conditions $a_0 = 2$ and $a_1 = 7$.
    \footnotesize
    \begin{enumerate}[<+->]
        \item Characteristic equation and its roots: \\
        $ x^2 - x - 2 = 0 $ \\
        $ (x + 1)(x - 2) = 0 $ \\
        So, $x_1 = 2$ and $x_2 = -1$.
        \item General form of the solution: \\
        $ a_n = \alpha_1 2^n + \alpha_2 (-1)^n $.
        \item Initial condition equations and their solutions: \\
        $ a_0 = \alpha_1 + \alpha_2 = 2 $ \\
        $ a_1 = \alpha_1 2 + \alpha_2 (-1) = 7 $ \\
        So, $\alpha_1 = 3$ and $\alpha_2 = -1$.
        \item Final answer: \\
        $ a_n = 3 \cdot 2^n - (-1)^n $ is a solution.
    \end{enumerate}
\end{frame}

\begin{frame}{Example 2}
    What is the solution of the recurrence relation $a_n = -a_{n-1} + 4 a_{n-2} + 4 a_{n-3}$ with initial conditions $a_0 = 8$, $a_1 = 6$ and $a_2 = 26$.
    \footnotesize
    \begin{enumerate}[<+->]
        \item Characteristic equation and its roots: \\
        $ x^3 + x^2 - 4x - 4 = 0 $ \\
        $ (x + 1)(x + 2)(x - 2) = 0 $ \\
        So, $x_1 = -1$, $x_2 = -2$ and $x_3 = 2$.
        \item General form of the solution: \\
        $ a_n = \alpha_1 (-1)^n + \alpha_2 (-2)^n + \alpha_3 2^n $.
        \item Initial condition equations and their solutions: \\
        $ a_0 = \alpha_1 + \alpha_2 + \alpha_3 = 8 $ \\
        $ a_1 = - \alpha_1 - 2 \alpha_2 + 2 \alpha_3 = 6 $ \\
        $ a_2 = \alpha_1 + 4 \alpha_2 + 4 \alpha_3 = 26 $ \\
        So, $\alpha_1 = 2$, $\alpha_2 = 1$ and $\alpha_3 = 5$.
        \item Final answer: \\
        $ a_n = 2 \cdot (-1)^n + (-2)^n + 5 \cdot 2^n $ is a solution.
    \end{enumerate}
\end{frame}

\end{document}
